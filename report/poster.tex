\documentclass[ % the name of the author
                    author={Callum Mann},
                % the degree programme
                % the dissertation    title (which cannot be blank)
                     title={Genetic algorithm for the CVRP},
                % the dissertation subtitle (which can    be blank)
                  subtitle={Capacitated Vehicle Routing Problem},
                % the dissertation     type
                      type={Heuristic},
                % the year of submission
                      year={2016}]{poster}

\begin{document}

% -----------------------------------------------------------------------------

\begin{frame}{}

% \begin{columns}[t]
%   \begin{column}{0.900\linewidth}
%   \begin{block}{\Large Capacitated Vehicle Routing Problem}
%     \vspace{10cm}
%   \end{block}
%   \end{column}
% \end{columns}

\begin{columns}[t]
  \begin{column}{0.422\linewidth}
    \begin{block}{\Large Representation}
      The representation of solutions (\textit{chromosomes}) are lists of tours that visit
      customers. The tour may exceed the trucks capacity making the solution infeasible,
      however there is a penalty term associated with each solution that affects their fitness.
      The benefit of this representation is that it allows genetic operators to arrange
      routes more freely and potentially generate much more optimal offsprings, as
      the optimal solutions are often very close to being infeasible.


    \end{block}
    \vspace{1cm}

    \begin{block}{\Large Initial Population}
      The initial population is designed to contain optimised routes with some randomness.
      The purpose of some randomness in the initial solution is so that the population does not
      converge too quickly resulting in cost stagnation. \vspace{1cm} \\
      The population is created by choosing a random starting node, and performing nearest neighbour
      until the capacity of the truck is reached. With some probability, the nearest neighbour will favour
      routes closer to the depot, so that the population is not identical at the beginning. After these routes have been
      created, they are passed into the \textbf{Steepest Improvement Algorithm (SIA)}, to optimise the order of the routes.
      These routes then have a starting cost of around 6500, depending on the randomness parameter.
    \end{block}
    \vspace{1cm}

    \vfill

    \begin{block}{\Large Generation Step}
      The population is changed every step, by removing an undesirable solution and
      replacing it with a new child that ideally has a better fitness value. The generation
      is advanced through the following steps:
      \begin{enumerate}
        \item P1, P2 = ParentSelection()
        \item Child = BiggestOverlapCrossover(P1, P2)
        \item Child = Iterate crossover(P1, Child) n times
        \item Mutate(Child)
        \item SteeplyImprove(Child)
        \item Repair(Child)
        \item Introduce(Child)
      \end{enumerate}
      \vspace{1cm}

      P1 is typically a solution with a high fitness value, and P2 is randomly chosen.
      This is done to maximise the chance of creating new optimal solutions. The child
      is created from a crossover in P1 and P2 that may be repeated several times between
      P1 and itself. This increases the chance of an improved child per iteration.
      \vspace{1cm} \\
      The child is also passed through the \textbf{SIA} after the operators as
      they often leave some room for local optimisation after routes have been
      moved around. The child is partially repaired at the end of the generation
      so that routes cannot become unsightly,
      although typically only feasible solutions are optimal in later generations
      due to the penalty term outweighing reduced cost.

    \end{block}
    \vspace{1cm}

  \end{column}

  \begin{column}{0.450\linewidth}
    \begin{block}{\Large Local Search: Steepest Improvement Algorithm $^{[0]}$}
      The order in which a truck will visit customers may be very unoptimised, so the \textbf{SIA} is used
      to search locally \textit{within} the solution for lower costs. The algorithm iterates over two indices,
      and compares the distance between themselves and the next customers. The customers are then swapped
      if the comparisons show a lesser cost between the indices. This process is repeated upon a route until there
      is no more cost to reduce.
    \end{block}
    \vspace{1cm}

    \vfill

    \begin{block}{\Large Crossover: Biggest Route Overlap $^{[0]}$}
      In this crossover a random subroute of one solution is placed into another
      solution. First, the bounding boxes for all routes are calculated, then
      the bounding boxes of the routes in the destination solutions are compared
      with the random subroute. Out of the routes with maximal bounding box overlap,
      the route with minimal demand is chosen for the subroute to be placed into optimally.
    \end{block}
    \vspace{1cm}

    \begin{block}{\Large Mutation: Minimal cost insertion}
      This mutation selects a random customer from the solution and places it into
      another route. The selected customer is inserted between the two customers
      that minimise distance travelled. This typically helps later in the
      algorithm when moving whole routes causes too much cost increase.
    \end{block}
    \vspace{1cm}

    \begin{block}{\Large Evaluation}
      The lowest cost every achieved by the algorithm was $5967$ for the \textit{fruitybun250}
      dataset, which represents a solution within $7\%$ of the best known value ($5583$). As with many GAs,
      the downfall is population convergence and stagnation, and other heuristics such as TABUROUTE$^{[1]}$
      can achieve $1\%$ within the optimal.
    \end{block}

    \begin{block}{\Large References}
      [0] - \textit{Solving the Vehicle Routing Problem with Genetic Algorithms}, \\ A. S. Bjarnadottir \\
      http://etd.dtu.dk/thesis/154736/imm3183.pdf \vspace{0.5cm} \\
      $[1]$ - \textit{A Tabu Search Heuristic For The Vehicle Routing Problem}, \\ M. Gendreau et. al
    \end{block}
    \vspace{1cm}
  \end{column}
\end{columns}

\vfill

\end{frame}

% -----------------------------------------------------------------------------

\end{document}
