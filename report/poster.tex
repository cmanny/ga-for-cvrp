\documentclass[ % the name of the author
                    author={Callum Mann},
                % the degree programme
                % the dissertation    title (which cannot be blank)
                     title={Genetic algorithms for the CVRP},
                % the dissertation subtitle (which can    be blank)
                  subtitle={Capacitated Vehicle Routing Problem},
                % the dissertation     type
                      type={Heuristics},
                % the year of submission
                      year={2016}]{poster}

\begin{document}

% -----------------------------------------------------------------------------

\begin{frame}{}

% \begin{columns}[t]
%   \begin{column}{0.900\linewidth}
%   \begin{block}{\Large Capacitated Vehicle Routing Problem}
%     \vspace{10cm}
%   \end{block}
%   \end{column}
% \end{columns}

\begin{columns}[t]
  \begin{column}{0.422\linewidth}
  \begin{block}{\Large Initial Population}
  The initial population is designed to contain optimised routes with some randomness. 
  The purpose of some randomness in the initial solution is so that the population does not
  converge too quickly resulting in cost stagnation. \\ \\
  The population is created by choosing a random starting node, and performing nearest neighbour
  until the capacity of the truck is reached. With some probability, the nearest neighbour will favour
  routes closer to the depot, so that the population is not identical at the beginning. After these routes have been
  created, they are passed into the \textbf{Steepest Improvement Algorithm}, to optimise the order of the routes.
  These routes then have a starting cost of around 6500, depending on the randomness parameter.
  \end{block}
  \end{column}

  \begin{column}{0.422\linewidth}
  \begin{block}{\Large Local Search: Steepest Improvement Algorithm}
    The order in which a truck will visit customers may be very unoptimised, so the \textbf{SIA} is used
    to search locally \textit{within} the solution for lower costs. 
  \end{block}
  \end{column}
\end{columns}

\vfill

\begin{columns}[t]
  \begin{column}{0.422\linewidth}
  \begin{block}{\Large Generation Step}
  The population is changed every step, by removing an undesirable solution and 
  replacing it with a new child that hopefully has a better fitness value. The generation
  is advanced through the following steps:
  \begin{enumerate}
    \item P1, P2 = ParentSelection()
    \item Child = BiggestOverlapCrossover(P1, P2)
    \item Child = Iterate crossover(P1, Child) n times
    \item Mutate(Child)
    \item Repair(Child)
    \item Introduce(Child)
  \end{enumerate}
  \end{block}
  \end{column}
  \begin{column}{0.422\linewidth}
  \begin{block}{\Large Biggest Overlap Crossover}
    \vspace{10cm}
  \end{block}
  \end{column}
\end{columns}

\vfill

\end{frame}

% -----------------------------------------------------------------------------

\end{document}
